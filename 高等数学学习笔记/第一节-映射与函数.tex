
% Default to the notebook output style

    


% Inherit from the specified cell style.




    
\documentclass[11pt]{article}

    
    
    \usepackage[T1]{fontenc}
    % Nicer default font (+ math font) than Computer Modern for most use cases
    \usepackage{mathpazo}

    % Basic figure setup, for now with no caption control since it's done
    % automatically by Pandoc (which extracts ![](path) syntax from Markdown).
    \usepackage{graphicx}
    % We will generate all images so they have a width \maxwidth. This means
    % that they will get their normal width if they fit onto the page, but
    % are scaled down if they would overflow the margins.
    \makeatletter
    \def\maxwidth{\ifdim\Gin@nat@width>\linewidth\linewidth
    \else\Gin@nat@width\fi}
    \makeatother
    \let\Oldincludegraphics\includegraphics
    % Set max figure width to be 80% of text width, for now hardcoded.
    \renewcommand{\includegraphics}[1]{\Oldincludegraphics[width=.8\maxwidth]{#1}}
    % Ensure that by default, figures have no caption (until we provide a
    % proper Figure object with a Caption API and a way to capture that
    % in the conversion process - todo).
    \usepackage{caption}
    \DeclareCaptionLabelFormat{nolabel}{}
    \captionsetup{labelformat=nolabel}

    \usepackage{adjustbox} % Used to constrain images to a maximum size 
    \usepackage{xcolor} % Allow colors to be defined
    \usepackage{enumerate} % Needed for markdown enumerations to work
    \usepackage{geometry} % Used to adjust the document margins
    \usepackage{amsmath} % Equations
    \usepackage{amssymb} % Equations
    \usepackage{textcomp} % defines textquotesingle
    % Hack from http://tex.stackexchange.com/a/47451/13684:
    \AtBeginDocument{%
        \def\PYZsq{\textquotesingle}% Upright quotes in Pygmentized code
    }
    \usepackage{upquote} % Upright quotes for verbatim code
    \usepackage{eurosym} % defines \euro
    \usepackage[mathletters]{ucs} % Extended unicode (utf-8) support
    \usepackage[utf8x]{inputenc} % Allow utf-8 characters in the tex document
    \usepackage{fancyvrb} % verbatim replacement that allows latex
    \usepackage{grffile} % extends the file name processing of package graphics 
                         % to support a larger range 
    % The hyperref package gives us a pdf with properly built
    % internal navigation ('pdf bookmarks' for the table of contents,
    % internal cross-reference links, web links for URLs, etc.)
    \usepackage{hyperref}
    \usepackage{longtable} % longtable support required by pandoc >1.10
    \usepackage{booktabs}  % table support for pandoc > 1.12.2
    \usepackage[inline]{enumitem} % IRkernel/repr support (it uses the enumerate* environment)
    \usepackage[normalem]{ulem} % ulem is needed to support strikethroughs (\sout)
                                % normalem makes italics be italics, not underlines
    \usepackage{mathrsfs}
    

    
    
    % Colors for the hyperref package
    \definecolor{urlcolor}{rgb}{0,.145,.698}
    \definecolor{linkcolor}{rgb}{.71,0.21,0.01}
    \definecolor{citecolor}{rgb}{.12,.54,.11}

    % ANSI colors
    \definecolor{ansi-black}{HTML}{3E424D}
    \definecolor{ansi-black-intense}{HTML}{282C36}
    \definecolor{ansi-red}{HTML}{E75C58}
    \definecolor{ansi-red-intense}{HTML}{B22B31}
    \definecolor{ansi-green}{HTML}{00A250}
    \definecolor{ansi-green-intense}{HTML}{007427}
    \definecolor{ansi-yellow}{HTML}{DDB62B}
    \definecolor{ansi-yellow-intense}{HTML}{B27D12}
    \definecolor{ansi-blue}{HTML}{208FFB}
    \definecolor{ansi-blue-intense}{HTML}{0065CA}
    \definecolor{ansi-magenta}{HTML}{D160C4}
    \definecolor{ansi-magenta-intense}{HTML}{A03196}
    \definecolor{ansi-cyan}{HTML}{60C6C8}
    \definecolor{ansi-cyan-intense}{HTML}{258F8F}
    \definecolor{ansi-white}{HTML}{C5C1B4}
    \definecolor{ansi-white-intense}{HTML}{A1A6B2}
    \definecolor{ansi-default-inverse-fg}{HTML}{FFFFFF}
    \definecolor{ansi-default-inverse-bg}{HTML}{000000}

    % commands and environments needed by pandoc snippets
    % extracted from the output of `pandoc -s`
    \providecommand{\tightlist}{%
      \setlength{\itemsep}{0pt}\setlength{\parskip}{0pt}}
    \DefineVerbatimEnvironment{Highlighting}{Verbatim}{commandchars=\\\{\}}
    % Add ',fontsize=\small' for more characters per line
    \newenvironment{Shaded}{}{}
    \newcommand{\KeywordTok}[1]{\textcolor[rgb]{0.00,0.44,0.13}{\textbf{{#1}}}}
    \newcommand{\DataTypeTok}[1]{\textcolor[rgb]{0.56,0.13,0.00}{{#1}}}
    \newcommand{\DecValTok}[1]{\textcolor[rgb]{0.25,0.63,0.44}{{#1}}}
    \newcommand{\BaseNTok}[1]{\textcolor[rgb]{0.25,0.63,0.44}{{#1}}}
    \newcommand{\FloatTok}[1]{\textcolor[rgb]{0.25,0.63,0.44}{{#1}}}
    \newcommand{\CharTok}[1]{\textcolor[rgb]{0.25,0.44,0.63}{{#1}}}
    \newcommand{\StringTok}[1]{\textcolor[rgb]{0.25,0.44,0.63}{{#1}}}
    \newcommand{\CommentTok}[1]{\textcolor[rgb]{0.38,0.63,0.69}{\textit{{#1}}}}
    \newcommand{\OtherTok}[1]{\textcolor[rgb]{0.00,0.44,0.13}{{#1}}}
    \newcommand{\AlertTok}[1]{\textcolor[rgb]{1.00,0.00,0.00}{\textbf{{#1}}}}
    \newcommand{\FunctionTok}[1]{\textcolor[rgb]{0.02,0.16,0.49}{{#1}}}
    \newcommand{\RegionMarkerTok}[1]{{#1}}
    \newcommand{\ErrorTok}[1]{\textcolor[rgb]{1.00,0.00,0.00}{\textbf{{#1}}}}
    \newcommand{\NormalTok}[1]{{#1}}
    
    % Additional commands for more recent versions of Pandoc
    \newcommand{\ConstantTok}[1]{\textcolor[rgb]{0.53,0.00,0.00}{{#1}}}
    \newcommand{\SpecialCharTok}[1]{\textcolor[rgb]{0.25,0.44,0.63}{{#1}}}
    \newcommand{\VerbatimStringTok}[1]{\textcolor[rgb]{0.25,0.44,0.63}{{#1}}}
    \newcommand{\SpecialStringTok}[1]{\textcolor[rgb]{0.73,0.40,0.53}{{#1}}}
    \newcommand{\ImportTok}[1]{{#1}}
    \newcommand{\DocumentationTok}[1]{\textcolor[rgb]{0.73,0.13,0.13}{\textit{{#1}}}}
    \newcommand{\AnnotationTok}[1]{\textcolor[rgb]{0.38,0.63,0.69}{\textbf{\textit{{#1}}}}}
    \newcommand{\CommentVarTok}[1]{\textcolor[rgb]{0.38,0.63,0.69}{\textbf{\textit{{#1}}}}}
    \newcommand{\VariableTok}[1]{\textcolor[rgb]{0.10,0.09,0.49}{{#1}}}
    \newcommand{\ControlFlowTok}[1]{\textcolor[rgb]{0.00,0.44,0.13}{\textbf{{#1}}}}
    \newcommand{\OperatorTok}[1]{\textcolor[rgb]{0.40,0.40,0.40}{{#1}}}
    \newcommand{\BuiltInTok}[1]{{#1}}
    \newcommand{\ExtensionTok}[1]{{#1}}
    \newcommand{\PreprocessorTok}[1]{\textcolor[rgb]{0.74,0.48,0.00}{{#1}}}
    \newcommand{\AttributeTok}[1]{\textcolor[rgb]{0.49,0.56,0.16}{{#1}}}
    \newcommand{\InformationTok}[1]{\textcolor[rgb]{0.38,0.63,0.69}{\textbf{\textit{{#1}}}}}
    \newcommand{\WarningTok}[1]{\textcolor[rgb]{0.38,0.63,0.69}{\textbf{\textit{{#1}}}}}
    
    
    % Define a nice break command that doesn't care if a line doesn't already
    % exist.
    \def\br{\hspace*{\fill} \\* }
    % Math Jax compatibility definitions
    \def\gt{>}
    \def\lt{<}
    \let\Oldtex\TeX
    \let\Oldlatex\LaTeX
    \renewcommand{\TeX}{\textrm{\Oldtex}}
    \renewcommand{\LaTeX}{\textrm{\Oldlatex}}
    % Document parameters
    % Document title
    \title{???-?????}
    
    
    
    
    

    % Pygments definitions
    
\makeatletter
\def\PY@reset{\let\PY@it=\relax \let\PY@bf=\relax%
    \let\PY@ul=\relax \let\PY@tc=\relax%
    \let\PY@bc=\relax \let\PY@ff=\relax}
\def\PY@tok#1{\csname PY@tok@#1\endcsname}
\def\PY@toks#1+{\ifx\relax#1\empty\else%
    \PY@tok{#1}\expandafter\PY@toks\fi}
\def\PY@do#1{\PY@bc{\PY@tc{\PY@ul{%
    \PY@it{\PY@bf{\PY@ff{#1}}}}}}}
\def\PY#1#2{\PY@reset\PY@toks#1+\relax+\PY@do{#2}}

\expandafter\def\csname PY@tok@w\endcsname{\def\PY@tc##1{\textcolor[rgb]{0.73,0.73,0.73}{##1}}}
\expandafter\def\csname PY@tok@c\endcsname{\let\PY@it=\textit\def\PY@tc##1{\textcolor[rgb]{0.25,0.50,0.50}{##1}}}
\expandafter\def\csname PY@tok@cp\endcsname{\def\PY@tc##1{\textcolor[rgb]{0.74,0.48,0.00}{##1}}}
\expandafter\def\csname PY@tok@k\endcsname{\let\PY@bf=\textbf\def\PY@tc##1{\textcolor[rgb]{0.00,0.50,0.00}{##1}}}
\expandafter\def\csname PY@tok@kp\endcsname{\def\PY@tc##1{\textcolor[rgb]{0.00,0.50,0.00}{##1}}}
\expandafter\def\csname PY@tok@kt\endcsname{\def\PY@tc##1{\textcolor[rgb]{0.69,0.00,0.25}{##1}}}
\expandafter\def\csname PY@tok@o\endcsname{\def\PY@tc##1{\textcolor[rgb]{0.40,0.40,0.40}{##1}}}
\expandafter\def\csname PY@tok@ow\endcsname{\let\PY@bf=\textbf\def\PY@tc##1{\textcolor[rgb]{0.67,0.13,1.00}{##1}}}
\expandafter\def\csname PY@tok@nb\endcsname{\def\PY@tc##1{\textcolor[rgb]{0.00,0.50,0.00}{##1}}}
\expandafter\def\csname PY@tok@nf\endcsname{\def\PY@tc##1{\textcolor[rgb]{0.00,0.00,1.00}{##1}}}
\expandafter\def\csname PY@tok@nc\endcsname{\let\PY@bf=\textbf\def\PY@tc##1{\textcolor[rgb]{0.00,0.00,1.00}{##1}}}
\expandafter\def\csname PY@tok@nn\endcsname{\let\PY@bf=\textbf\def\PY@tc##1{\textcolor[rgb]{0.00,0.00,1.00}{##1}}}
\expandafter\def\csname PY@tok@ne\endcsname{\let\PY@bf=\textbf\def\PY@tc##1{\textcolor[rgb]{0.82,0.25,0.23}{##1}}}
\expandafter\def\csname PY@tok@nv\endcsname{\def\PY@tc##1{\textcolor[rgb]{0.10,0.09,0.49}{##1}}}
\expandafter\def\csname PY@tok@no\endcsname{\def\PY@tc##1{\textcolor[rgb]{0.53,0.00,0.00}{##1}}}
\expandafter\def\csname PY@tok@nl\endcsname{\def\PY@tc##1{\textcolor[rgb]{0.63,0.63,0.00}{##1}}}
\expandafter\def\csname PY@tok@ni\endcsname{\let\PY@bf=\textbf\def\PY@tc##1{\textcolor[rgb]{0.60,0.60,0.60}{##1}}}
\expandafter\def\csname PY@tok@na\endcsname{\def\PY@tc##1{\textcolor[rgb]{0.49,0.56,0.16}{##1}}}
\expandafter\def\csname PY@tok@nt\endcsname{\let\PY@bf=\textbf\def\PY@tc##1{\textcolor[rgb]{0.00,0.50,0.00}{##1}}}
\expandafter\def\csname PY@tok@nd\endcsname{\def\PY@tc##1{\textcolor[rgb]{0.67,0.13,1.00}{##1}}}
\expandafter\def\csname PY@tok@s\endcsname{\def\PY@tc##1{\textcolor[rgb]{0.73,0.13,0.13}{##1}}}
\expandafter\def\csname PY@tok@sd\endcsname{\let\PY@it=\textit\def\PY@tc##1{\textcolor[rgb]{0.73,0.13,0.13}{##1}}}
\expandafter\def\csname PY@tok@si\endcsname{\let\PY@bf=\textbf\def\PY@tc##1{\textcolor[rgb]{0.73,0.40,0.53}{##1}}}
\expandafter\def\csname PY@tok@se\endcsname{\let\PY@bf=\textbf\def\PY@tc##1{\textcolor[rgb]{0.73,0.40,0.13}{##1}}}
\expandafter\def\csname PY@tok@sr\endcsname{\def\PY@tc##1{\textcolor[rgb]{0.73,0.40,0.53}{##1}}}
\expandafter\def\csname PY@tok@ss\endcsname{\def\PY@tc##1{\textcolor[rgb]{0.10,0.09,0.49}{##1}}}
\expandafter\def\csname PY@tok@sx\endcsname{\def\PY@tc##1{\textcolor[rgb]{0.00,0.50,0.00}{##1}}}
\expandafter\def\csname PY@tok@m\endcsname{\def\PY@tc##1{\textcolor[rgb]{0.40,0.40,0.40}{##1}}}
\expandafter\def\csname PY@tok@gh\endcsname{\let\PY@bf=\textbf\def\PY@tc##1{\textcolor[rgb]{0.00,0.00,0.50}{##1}}}
\expandafter\def\csname PY@tok@gu\endcsname{\let\PY@bf=\textbf\def\PY@tc##1{\textcolor[rgb]{0.50,0.00,0.50}{##1}}}
\expandafter\def\csname PY@tok@gd\endcsname{\def\PY@tc##1{\textcolor[rgb]{0.63,0.00,0.00}{##1}}}
\expandafter\def\csname PY@tok@gi\endcsname{\def\PY@tc##1{\textcolor[rgb]{0.00,0.63,0.00}{##1}}}
\expandafter\def\csname PY@tok@gr\endcsname{\def\PY@tc##1{\textcolor[rgb]{1.00,0.00,0.00}{##1}}}
\expandafter\def\csname PY@tok@ge\endcsname{\let\PY@it=\textit}
\expandafter\def\csname PY@tok@gs\endcsname{\let\PY@bf=\textbf}
\expandafter\def\csname PY@tok@gp\endcsname{\let\PY@bf=\textbf\def\PY@tc##1{\textcolor[rgb]{0.00,0.00,0.50}{##1}}}
\expandafter\def\csname PY@tok@go\endcsname{\def\PY@tc##1{\textcolor[rgb]{0.53,0.53,0.53}{##1}}}
\expandafter\def\csname PY@tok@gt\endcsname{\def\PY@tc##1{\textcolor[rgb]{0.00,0.27,0.87}{##1}}}
\expandafter\def\csname PY@tok@err\endcsname{\def\PY@bc##1{\setlength{\fboxsep}{0pt}\fcolorbox[rgb]{1.00,0.00,0.00}{1,1,1}{\strut ##1}}}
\expandafter\def\csname PY@tok@kc\endcsname{\let\PY@bf=\textbf\def\PY@tc##1{\textcolor[rgb]{0.00,0.50,0.00}{##1}}}
\expandafter\def\csname PY@tok@kd\endcsname{\let\PY@bf=\textbf\def\PY@tc##1{\textcolor[rgb]{0.00,0.50,0.00}{##1}}}
\expandafter\def\csname PY@tok@kn\endcsname{\let\PY@bf=\textbf\def\PY@tc##1{\textcolor[rgb]{0.00,0.50,0.00}{##1}}}
\expandafter\def\csname PY@tok@kr\endcsname{\let\PY@bf=\textbf\def\PY@tc##1{\textcolor[rgb]{0.00,0.50,0.00}{##1}}}
\expandafter\def\csname PY@tok@bp\endcsname{\def\PY@tc##1{\textcolor[rgb]{0.00,0.50,0.00}{##1}}}
\expandafter\def\csname PY@tok@fm\endcsname{\def\PY@tc##1{\textcolor[rgb]{0.00,0.00,1.00}{##1}}}
\expandafter\def\csname PY@tok@vc\endcsname{\def\PY@tc##1{\textcolor[rgb]{0.10,0.09,0.49}{##1}}}
\expandafter\def\csname PY@tok@vg\endcsname{\def\PY@tc##1{\textcolor[rgb]{0.10,0.09,0.49}{##1}}}
\expandafter\def\csname PY@tok@vi\endcsname{\def\PY@tc##1{\textcolor[rgb]{0.10,0.09,0.49}{##1}}}
\expandafter\def\csname PY@tok@vm\endcsname{\def\PY@tc##1{\textcolor[rgb]{0.10,0.09,0.49}{##1}}}
\expandafter\def\csname PY@tok@sa\endcsname{\def\PY@tc##1{\textcolor[rgb]{0.73,0.13,0.13}{##1}}}
\expandafter\def\csname PY@tok@sb\endcsname{\def\PY@tc##1{\textcolor[rgb]{0.73,0.13,0.13}{##1}}}
\expandafter\def\csname PY@tok@sc\endcsname{\def\PY@tc##1{\textcolor[rgb]{0.73,0.13,0.13}{##1}}}
\expandafter\def\csname PY@tok@dl\endcsname{\def\PY@tc##1{\textcolor[rgb]{0.73,0.13,0.13}{##1}}}
\expandafter\def\csname PY@tok@s2\endcsname{\def\PY@tc##1{\textcolor[rgb]{0.73,0.13,0.13}{##1}}}
\expandafter\def\csname PY@tok@sh\endcsname{\def\PY@tc##1{\textcolor[rgb]{0.73,0.13,0.13}{##1}}}
\expandafter\def\csname PY@tok@s1\endcsname{\def\PY@tc##1{\textcolor[rgb]{0.73,0.13,0.13}{##1}}}
\expandafter\def\csname PY@tok@mb\endcsname{\def\PY@tc##1{\textcolor[rgb]{0.40,0.40,0.40}{##1}}}
\expandafter\def\csname PY@tok@mf\endcsname{\def\PY@tc##1{\textcolor[rgb]{0.40,0.40,0.40}{##1}}}
\expandafter\def\csname PY@tok@mh\endcsname{\def\PY@tc##1{\textcolor[rgb]{0.40,0.40,0.40}{##1}}}
\expandafter\def\csname PY@tok@mi\endcsname{\def\PY@tc##1{\textcolor[rgb]{0.40,0.40,0.40}{##1}}}
\expandafter\def\csname PY@tok@il\endcsname{\def\PY@tc##1{\textcolor[rgb]{0.40,0.40,0.40}{##1}}}
\expandafter\def\csname PY@tok@mo\endcsname{\def\PY@tc##1{\textcolor[rgb]{0.40,0.40,0.40}{##1}}}
\expandafter\def\csname PY@tok@ch\endcsname{\let\PY@it=\textit\def\PY@tc##1{\textcolor[rgb]{0.25,0.50,0.50}{##1}}}
\expandafter\def\csname PY@tok@cm\endcsname{\let\PY@it=\textit\def\PY@tc##1{\textcolor[rgb]{0.25,0.50,0.50}{##1}}}
\expandafter\def\csname PY@tok@cpf\endcsname{\let\PY@it=\textit\def\PY@tc##1{\textcolor[rgb]{0.25,0.50,0.50}{##1}}}
\expandafter\def\csname PY@tok@c1\endcsname{\let\PY@it=\textit\def\PY@tc##1{\textcolor[rgb]{0.25,0.50,0.50}{##1}}}
\expandafter\def\csname PY@tok@cs\endcsname{\let\PY@it=\textit\def\PY@tc##1{\textcolor[rgb]{0.25,0.50,0.50}{##1}}}

\def\PYZbs{\char`\\}
\def\PYZus{\char`\_}
\def\PYZob{\char`\{}
\def\PYZcb{\char`\}}
\def\PYZca{\char`\^}
\def\PYZam{\char`\&}
\def\PYZlt{\char`\<}
\def\PYZgt{\char`\>}
\def\PYZsh{\char`\#}
\def\PYZpc{\char`\%}
\def\PYZdl{\char`\$}
\def\PYZhy{\char`\-}
\def\PYZsq{\char`\'}
\def\PYZdq{\char`\"}
\def\PYZti{\char`\~}
% for compatibility with earlier versions
\def\PYZat{@}
\def\PYZlb{[}
\def\PYZrb{]}
\makeatother


    % Exact colors from NB
    \definecolor{incolor}{rgb}{0.0, 0.0, 0.5}
    \definecolor{outcolor}{rgb}{0.545, 0.0, 0.0}



    
    % Prevent overflowing lines due to hard-to-break entities
    \sloppy 
    % Setup hyperref package
    \hypersetup{
      breaklinks=true,  % so long urls are correctly broken across lines
      colorlinks=true,
      urlcolor=urlcolor,
      linkcolor=linkcolor,
      citecolor=citecolor,
      }
    % Slightly bigger margins than the latex defaults
    
    \geometry{verbose,tmargin=1in,bmargin=1in,lmargin=1in,rmargin=1in}
    
    

    \begin{document}
    
    
    \maketitle
    
    

    
    \section{集合}\label{ux96c6ux5408}

\subsection{定义}\label{ux5b9aux4e49}

集合(集)是指具有某种特定性质的事物的总体,组成这个集合的事物称为该集合的元素(元)。通常用大写拉丁字母A,
B, C\ldots{}\ldots{}表示集合,用小写拉丁字母a, b,
c\ldots{}\ldots{}表示集合中的元素。 \#\# 集合与元素的关系
如果a是集合A的元素,就说a属于A,记作a\(\in\)A;如果a不是集合或的元素,就说a不属于A,记作a\(\notin\)A。
\#\# 集合的种类
若它只含有限个元素,则称为有限集;不是有限集的集合称为无限集。 \#\#
集合的表示法 \#\#\# 列举法
把集合的全体元素一一列举出来表示,例如,由元素\(a_{1}, a_{2}, \cdots, a_{n}\)组成的集合A,可表示成\(A=\left\{a_{1}, a_{2}, \cdots, a_{n}\right\}\)。
\#\#\# 描述法
若集合M是由具有某种性质P的元素x的全体所组成的,就可表示成\(M=\{x | x 具有性质P\}\)。如集合B是方程\(x^{2}-1=0\)的解集,就可表示成\(B=\left\{x | x^{2}-1=0\right\}\)
\#\# 几种常见数集
对于数集,有时我们在表示数集的字母的右上角标上''*''来排除0的集,标上''+``来表示该数集内排除0与负数
- 全体非负整数(自然数) \(N=\{0,1,2, \cdots, n, \cdots\}\) - 全体正整数
\(N^{+}=\{1,2,3, \cdots, n, \cdots\}\) - 全体整数
\(\mathrm{Z}=\{\cdots,-n, \cdots,-2,-1,0,1,2, \cdots, n, \cdots\}\) -
全体有理数
\(\mathbf{Q}=\left\{\frac{p}{q} | p \in \mathbf{Z}, q \in \mathbf{N}^{+}\right. 且p与q互质\}\)
- 全体实数 \(R\)

\subsection{几种特殊集合}\label{ux51e0ux79cdux7279ux6b8aux96c6ux5408}

\subsubsection{子集}\label{ux5b50ux96c6}

设A、B是两个集合,如果集合A的元素都是集合B的元素,则称A是B的\_\_子集\_\_,记作\(A \subset B\)(读作A包含于B),或\(B \supset A\)(读作B包含A)。
如果集合A与集合B互为子集,即\(A \subset B\)且\(B \subset A\),则称集合A与集合B
\textbf{相等},记作\(A = B\),例如,设

\(A=\{1,2\}, \quad B=\{x| x^{2}-3 x+2=0 \}\)

则\(A = B\)

若\(A \subset B\)且\(A \neq B\),即称A是B的\_\_真子集\_\_,记作\(A \nsubseteq B\),例如,\(N \leqq Z \leqq Q \leqq R\).
\#\#\# 空集 不含任何元素的集合称为\_\_空集\_\_.例如

\(\{x | x \in \mathbf{R}且x^{2}+1=0 \}\)

是空集,因为适合条件\(\{x^{2}+1=0 \}\)的实数是不存在的.空集记作\(\varnothing\),且规定空集\(\varnothing\)是任何集合A的子集,即\(\varnothing \subset A\).

\subsection{集合的运算}\label{ux96c6ux5408ux7684ux8fd0ux7b97}

\subsubsection{并集}\label{ux5e76ux96c6}

设A,B是两个集合,由所有属于A或者B的元素组成的集合,称为A与B的并集(简称并),记作\(A \cup B\),即

\(A \cup B=\{x| x \in A或 x \in B \}\)

\subsubsection{交集}\label{ux4ea4ux96c6}

由所有既属于A又属于B的元素组成的集合,称为A与B的交集(简称交),记作\(A \cap B\),即

\(A \cap B=\{x| x \in A且 x \in B \}\)

\subsubsection{差集}\label{ux5deeux96c6}

由所有属于A而不属于B的元素组成的集合,称为A与B的差集(简称差),记作\(A \backslash B\),即

\(A \backslash B=\{x| x \in A且 x \in B \}\)

\subsubsection{全集和补集}\label{ux5168ux96c6ux548cux8865ux96c6}

有时,我们研究某个问题限定在一个大的集合\(I\)中进行,所研究的其它集合\(A\)都是\(I\)的子集,此时,我们称集合\(I\)为\_\_全集\_\_或\_\_基本集\_\_,称\(I\backslash A\)为\(A\)的\_\_余集\_\_或\_\_补集\_\_,记作\(A^{C}\).

\subsubsection{运算法则}\label{ux8fd0ux7b97ux6cd5ux5219}

\paragraph{交换律}\label{ux4ea4ux6362ux5f8b}

\(A \cup B=B \cup A, \quad A \cap B=B \cap A\)

证明:

\((1)\quad x \in(A \cup B) \Rightarrow x \in A 或 x \in B \Rightarrow x \in(B \cup A) \Rightarrow(A \cup B) \subset(B \cup A)\)

\((2)\quad x \in(B \cup A) \Rightarrow x \in B 或 x \in A \Rightarrow x \in(A \cup B) \Rightarrow(B \cup A) \subset(A \cup B)\)

由相等的定义可知,

\(A \cup B=B \cup A\)

同理可证\(A \cap B=B \cap A\)

\paragraph{结合律}\label{ux7ed3ux5408ux5f8b}

\((A \cup B) \cup C=A \cup(B \cup C), (A \cap B) \cap C=A \cap(B \cap C)\)

证明:

\$(1)\quad x \in(A \cap B) \cap C \Rightarrow x \in  A 且 x \in B 且 x
\in C \Rightarrow x \in A \cap (B \cap C) \Rightarrow (A \cap B) \cap C
\subset A \cap (B \cap C) \(</center> <center>\)(2)\quad x \in A \cap (B
\cap C) \Rightarrow x \in  A 且 x \in B 且 x \in C \Rightarrow x \in (A
\cap B) \cap C \Rightarrow A \cap (B \cap C) \subset (A \cap B) \cap C
\(</center> 由相等的定义可知, <center>\)(A \cap B) \cap C=A \cap(B
\cap C)\$

同理可证\((A \cup B) \cup C=A \cup(B \cup C)\)

\paragraph{分配律}\label{ux5206ux914dux5f8b}

\((A \cup B) \cap C=(A \cap C) \cup(B \cap C), (A \cap B) \cup C=(A \cup C) \cap(B \cup C)\)

证明\((A \cap B) \cup C=(A \cup C) \cap(B \cup C)\):

\begin{enumerate}
\def\labelenumi{(\arabic{enumi})}
\tightlist
\item
  先证明 \$x \in(A \cap B) \cup C \$ 是
  \((A \cup C) \cap(B \cup C)\)的子集
\end{enumerate}

\$x \in(A \cap B) \cup C \Rightarrow x \in  A \cap B 或 C \$

\$\because x \in A \cap B 时 \Rightarrow x \in A 且 x \in B
\Rightarrow x \in A \cup C 且 x \in B \cup C \$

\$ 当 x \in C 时 \Rightarrow x \in C \cup A 且 x \in C \cup B
\Rightarrow x \in A \cup C 且 x \in B \cup C \$

\(\therefore (A \cap B) \cup C \subset (A \cup C) \cap(B \cup C)\)

\begin{enumerate}
\def\labelenumi{(\arabic{enumi})}
\setcounter{enumi}{1}
\tightlist
\item
  再证明\((A \cup C) \cap(B \cup C)\) 是 \$(A \cap B) \cup C \$的子集
\end{enumerate}

\$x \in(A \cup C) \cap(B \cup C) \Rightarrow x \in  A \cup C 且 x \in B
\cup C \$

\(若 x \in(A \cup C) \cap(B \cup C) 可能的情况有以下几种:\) - \$ x \in C
\Rightarrow x \in (A \cap B) \cup C \$ - \$ x \in A \cap B \Rightarrow x
\in (A \cap B) \cup C \$

其它情况均无法满足\(x \in(A \cup C) \cap(B \cup C)\)

所有满足\((A \cup C) \cap(B \cup C)\) 的情况均能推出\$x \in (A \cap B)
\cup C \$

\$\therefore (A \cup C) \cap(B \cup C)是 (A \cap B) \cup C \$的子集

由 (1)和(2)得出\((A \cap B) \cup C=(A \cup C) \cap(B \cup C)\)

可以用类似的方法证明\((A \cup B) \cap C=(A \cap C) \cup(B \cap C)\)

\paragraph{对偶律}\label{ux5bf9ux5076ux5f8b}

\((A \cup B)^{c}=A^{c} \cap B^{c}, (A \cap B)^{c}=A^{\mathrm{C}} \cup B^{c}\)

证明:

\(x \in(A \cup B)^{c} \Rightarrow x \notin A \cup B \Rightarrow x \notin A 且 x \notin B \Rightarrow x \in A^{c} 且 x \Rightarrow x \in A^{c} \cap B^{c}\)

\subsection{直积(笛卡尔乘积)}\label{ux76f4ux79efux7b1bux5361ux5c14ux4e58ux79ef}

设A, B是任意两个集合,在集合A中任意取一个元素x, 在集合B中任意取一个元素y,
组成一个有序对(x, y), 把这样的有序对作为新的元素,
它们全体组成的集合称为集合A与集合B的直积, 记为 \(A \times B\) 即
\(A \times B=\{(x, y) | x \in A 且 y \in B\}\) \#\# 区间 \#\#\# 有限区间
设a和b都是实数, 且\(\{a \lt b\}\). 数集 \(\{x|a \lt x \lt b\}\)
称为\_\_开区间\_\_, 记作\((a,b)\), 即

\((a,b)=\{x|a \lt x \lt b\}\)

a和b称为开区间\((a,b)\)的\_\_端点\_\_, 这里 \(a \notin (a,b)\),
\(b \notin (a,b)\).

数集\(\{x | a \leqslant x \leqslant b\}\) 称为\_\_闭区间\_\_,
记作\([a,b]\), 即

\([a,b] = \{x|a \leqslant x \leqslant b\}\)

a和b也称为闭区间\([a, b]\)的端点, 这里\(a \in [a, b], b \in [a, b]\).

类似地可说明:\([a, b)=\{x | a \leqslant x<b\} , (a, b]=\{x | a<x \leqslant b\}\)

\([a, b) 和(a, b]\) 都称为\_\_半开区间\_\_

\(b - a\)称为\_\_区间的长度\_\_ \#\#\# 无限区间
\([a,+\infty)=\{x | x \geqslant a\}, (-\infty, b)=\{x | x<b\}\)

\subsection{邻域}\label{ux90bbux57df}

以点a为中心的任何开区间称为点a的\_\_邻域\_\_,记作\(U(a)\)

设\(\delta\)是任意一个正数,则开区间\((a - \delta, a+ \delta)\)就是点a的一个邻域,
这个邻域称为点a的\(\delta邻域\), 记作\(U(a, \delta)\),
点a称为这个\_\_邻域的中心\_\_, \(\delta\)称为这个\_\_邻域的半径\_\_.
可写成

\(U(a, \delta)=\{x||x-a|<\delta \}\)

\subsubsection{去心邻域}\label{ux53bbux5fc3ux90bbux57df}

有时用到的邻域需要把邻域中心去掉,称为点a的去心\(\delta\)邻域\(U(a, \delta)\),
记作, 即

\(U(a, \delta)=\{x| 0<|x-a|<\delta \}\)

为了方便, 有时把开区间\((a - \delta, a)\)称为a的左\(\delta\)邻域,
把开区间\((a, a+ \delta)\)称为a的右\(\delta\)邻域.

\section{映射}\label{ux6620ux5c04}

设\(X, Y\)是两个非空集合,如果存在一个法则\(f\),
使得对\(X\)中每个元素\(x\),
按法则\(f\),在\(Y\)中有唯一确定的元素\(y\)与之对应,则称\(f\)为从\(X\)到\(Y\)的映射,
记作

\(f : X \rightarrow Y\)

其中\(y\)称为元素\(x\)(在映射\(f\)下)的\_\_像\_\_, 并记作 \(f(x)\), 即

\(y = f(x)\)

元素\(x\)称为元素\(y\)的\_\_原像\_\_; 集合\(X\)称为映射\(f\)的定义域,
记作\(D_{f}\), 即 \(D_{f}=X\);
\(X\)中所有元素的像所组成的集合称为映射\(f\)的值域,
记作\(R_{f}\)或\(f(X)\), 即

\(R_{f}=f(X)=\{f(x) | x \in X\}\)

\begin{itemize}
\tightlist
\item
  映射必须具备三个要素
\item
  集合\(X\), 即定义域\(D_{f} = X\)
\item
  集合\(Y\), 即值域的范围: \(R_{f} \subset Y\)
\item
  对应法则\(f\), 使对每个\(x \in X\) 有唯一确定的\(y = f(x)\) 与之对应
\item
  对于每个\(x \in X\), 元素\(x\)的像\(y\)是唯一的; 而对每个\(y \in R\),
  元素\(y\)的原像\(x\)不一定是唯一的;映射\(f\)的值域\(R_{f}\)是\(Y\)的一个子集,
  即\(R_{f} \subset Y\), 不一定\(R_{f} = Y\)
\end{itemize}

\subsection{满射 单射
双射}\label{ux6ee1ux5c04-ux5355ux5c04-ux53ccux5c04}

设\(f\)是从集合\(X\)到\(Y\)的映射, 若\(R_{f} = Y\),
即\(Y\)中任一元素\(y\)都是\(X\)中某元素的像,则称\(f\)为\(X\)到\(Y\)上的映射或\_\_满射\_\_.
若对\(X\)中任意两个不同元素\(x_{1} \ne x_{2}\),
它们的像\(f(x_{1}) \ne f(x_{2})\),
则称\(f\)为\(X\)到\(Y\)的\_\_单射\_\_;若映射\(f\)即是单射, 又是满射,
则称\(f\)为一一映射(或\_\_双射\_\_).

\subsection{逆映射}\label{ux9006ux6620ux5c04}

设\(f\)是\(X\)到\(Y\)的单射, 则由定义, 对每个\(y \in R_{f}\),
有唯一的\(x \in X\), 适合\(f(x) = y\). 于是,
我们可以定义一个\(R_{f}\)到\(X\)的新映射\(g\),
即\(g: R_{f} \rightarrow X\), 对每个\(y \in R_{f}\), 规定\(g(y) = x\),
这\(x\)满足\(f(x) = y\). 这个映射\(g\)称为\(f\)的\_\_逆映射\_\_,
记作\(f^{-1}\), 其定义域\(D_{f^{-1}}=R\), 值域\(R_{f^{-1}} = X\).

\subsection{复合映射}\label{ux590dux5408ux6620ux5c04}

设有两个映射

\(g: X \rightarrow Y_{1}, f: Y_{2} \rightarrow Z\)

其中\(Y_{1} \subset Y_{2}\).
则由映射\(g\)和\(f\)可以定出一个从\(X\)到\(Z\)的对应法则,它将每个\(x \in X\)映射成\(f[g(x)] \in Z\).
显然, 这个对应法则确定了一个从\(X\)到\(Z\)的映射,
这个映射称为映射\(g\)到\(f\)的复合映射, 记作\(f\circ g\), 即

\$f\circ g: X \rightarrow Z \(</center> <center>\)(f\circ g)(x)=
f{[}g(x){]}, x \in X \$

\section{函数}\label{ux51fdux6570}

设数集\(D \subset R\),
则称映射\(f: D \rightarrow R\)为定义在\(D\)上的函数, 通常简记为

\(y=f(x), x \in D\)

其中\(x\)称为\_\_自变量\_\_, \(y\)称为\_\_因变量\_\_, \(D\)称为定义域,
记作\(D_{f}\), 即\(D_{f} = D\), 函数值 \(f(x)\)
的全体所构成的集合称为函数\(f\)的\_\_值域\_\_, 记作\(R_{f} 或 f(D)\), 即

\(R_{f} = f(D) = \{y|y=f(x), x \in D\}\)

\(f\) 为对应法则, \(f(x)\)为自变量\(x\)的函数值,
但通常习惯使用记号\(f(x), x \in D 或 y=f(x), x \in D\)来表示定义在D上的函数,
这时候应理解 为由它所确定的函数\(f\).

函数是从实数集到实数集的映射, 其值域总在\(R\)内, 因此构成函数的要素是:
定义域\(D\)及对应法则\(f\), 如果两个函数的定义域相同, 对应法则也相同,
那么这两个函数就是相同的, 否则就是不同的.

    \begin{Verbatim}[commandchars=\\\{\}]
{\color{incolor}In [{\color{incolor}2}]:} \PY{k+kn}{import} \PY{n+nn}{sys}
        \PY{n}{sys}\PY{o}{.}\PY{n}{path}
\end{Verbatim}

\begin{Verbatim}[commandchars=\\\{\}]
{\color{outcolor}Out[{\color{outcolor}2}]:} ['/usr/lib/python36.zip',
         '/usr/lib/python3.6',
         '/usr/lib/python3.6/lib-dynload',
         '',
         '/home/learn/.local/lib/python3.6/site-packages',
         '/usr/local/lib/python3.6/dist-packages',
         '/home/learn/blog/src',
         '/usr/lib/python3/dist-packages',
         '/home/learn/.local/lib/python3.6/site-packages/IPython/extensions',
         '/home/learn/.ipython']
\end{Verbatim}
            
    \begin{Verbatim}[commandchars=\\\{\}]
{\color{incolor}In [{\color{incolor} }]:} 
\end{Verbatim}


    % Add a bibliography block to the postdoc
    
    
    
    \end{document}
